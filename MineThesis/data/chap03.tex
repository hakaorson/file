\chapter{复合物筛选模型设计}
\label{chapter:gcnfilter}

\section{相关算法介绍}
\label{section:arithmetic}
\subsection{网络嵌入}
\label{subsection:nodeEmbedding}
\subsection{图卷积神经网络}
\label{subsection:GCN}
\subsection{图自编码器}
\label{subsection:GAE}
\section{算法总体流程}
\label{section:progress}
\section{基于生物特征的模型}
\label{section:biofeatBaseModel}
可添加的特征包括蛋白质功能注释特征、结构域特征、亚细胞定位特征以及网络拓扑特征。参照\ref{section:featEngineer}介绍的特征提取方法,$PIN$中每条相互作用边所具有的特征维度为{},具体分布如表\ref{tab:PINedgeFeatNUms}所示:
\begin{table}[h]
    \centering
    \caption{$PIN$边特征维度分布}
    \label{tab:PINedgeFeatNUms}
    \begin{tabular}{C{3cm}C{3cm}C{3cm}C{3cm}}
        \toprule
        \textbf{功能注释特征} & \textbf{结构域特征} & \textbf{亚细胞定位特征} & \textbf{网络拓扑特征} \\
        \midrule
        2                     & 7                   & 2                       & 2                     \\
        \bottomrule
    \end{tabular}
\end{table}
生物上的,拓扑上的,异常数据处理,过大数据归一化
\section{基于全局特征的模型}
\label{section:GlobalfeatBaseModel}
GAE怎么做的,deepwalk怎么做的,node2vec怎么做的
\section{特征融合模型}
\label{section:fusionfeatBaseModel}
考虑少量蛋白质特征,考虑图拓扑特征
\section{尺寸敏感性的图读出算法}
\label{section:GPool}
