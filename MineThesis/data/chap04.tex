\chapter{实验结果与分析}
\label{chapter:resultAndOther}
两个网络上计算,因为比较经典
额外的三个方法测试,
只有最后最优模型才在四个网络上全跑,但是只是写一个同时,然后放一个实验结果说明有效就可以了
\section{相关参数和评价指标}
\label{section:metrix}

基于第


数据集分割等等
需要告诉算法的一般性设计,包括损失函数,batch等等

分类模型常用指标,precision,recall、fi。macro等等

复合物上的评价指标。。。


\begin{equation}
    \label{equ:compComplexSim:NA}
    NA_{(\mathcal{P} ,\mathcal{Q} )} = \frac{{\left\lvert V_{\mathcal{P}} \cap V_{\mathcal{Q}}\right\rvert}^2 }{{\left\lvert V_{\mathcal{P}} \right\rvert}\cdot  {\left\lvert V_{\mathcal{Q}} \right\rvert}} 
\end{equation}
其中$\mathcal{P}=(V_{\mathcal{P}} ,E_{\mathcal{P}})$是预测的复合物,$V_{\mathcal{P}}$为复合物中的蛋白质,$E_{\mathcal{P}}$为复合物中的相互作用,$\mathcal{Q}=(V_{\mathcal{Q}} ,E_{\mathcal{Q}})$是标准的复合物。

\section{生物特征的影响}E
\label{section:biofeatAnasys}
先比较边卷积模型和图特征的模型,在分类问题上的表现
然后是边卷积模型,融合特征边卷积模型总体的表现


两个网络上计算
额外的三个
\section{全局特征的影响}
\label{section:globalStructfeatAnasys}
两个网络上计算
\section{特征融合模型的实验结果分析}
\label{section:fusionfeatAnasys}
同时加上部分蛋白质的生物特征

\section{图读出算法的影响}
\label{section:gpoolAnasys}
得出最佳模型,并和刘晓霞做比较,由于什么的限制,只能在我的数据上跑刘晓霞的方法

\section{实验总结与分析}
\label{section:resultSummary}