\tongjisetup{
  %******************************
  % 注意:
  %   1. 配置里面不要出现空行
  %   2. 不需要的配置信息可以删除
  %******************************
  %
  %=====
  % 秘级
  %=====
  secretlevel={保密},
  secretyear={2},
  % doctor={0}
  %
  %=========
  % 中文信息
  %=========
  % 题目过长可以换行(推荐手动加入换行符,这样就可以控制换行的地方啦)。
  ctitle={蛋白质复合物筛选模型研究},
  cheadingtitle={蛋白质复合物筛选模型研究},    %用于页眉的标题,不要换行
  cauthor={高盼},
  studentnumber={1830801},
  cmajorfirst={工学},
  cmajorsecond={计算机科学与技术},
  cdepartment={电子与信息工程学院},
  csupervisor={关佶红 教授},
  % 如果没有副指导老师或者校外指导老师,把{}中内容留空即可,或者直接注释掉。
  % cassosupervisor={裴刚 教授~(校外)}, % 副指导老师
  % 日期自动使用当前时间,若需手动指定,按如下方式修改:
  % cdate={\zhdigits{2018}年\zhnumber{11}月},
  % 没有基金的话就注释掉吧。
  % cfunds={(本论文由我要努力想办法撑到两行的著名国家杰出青年基金 (No.123456789) 支持)},
  %
  %=========
  % 英文信息
  %=========
  etitle={English Title},
  eauthor={GAO PAN},
  emajorfirst={TODO},
  emajorsecond={TODO},
  edepartment={TODO},
  % 日期自动使用当前时间,若需手动指定,按如下方式修改:
  % edate={November,\ 2018},
  % efunds={(Supported by the Natural Science Foundation of China for\\ Distinguished Young Scholars, Grant No.123456789)},    
  esupervisor={TODO}
  % eassosupervisor={Prof. Gang Pei (XiaoWai)}
  % }
}
% 定义中英文摘要和关键字
\begin{cabstract}
  蛋白质复合物是蛋白质相互结合,完成某一项生物功能的集合。生物学上蛋白质复合物的发现与研究对细胞组成、药物发现等研究至关重要。蛋白质复合物可以通过生物实验的方法发现,但是生物实验成本较高、周期较长,无法满足后续的研究需求。
  
  随着计算理论的发展,蛋白质复合物发现邻域出现了基于计算的方法,将蛋白质之间广泛的相互作用抽象成图结构,蛋白质复合物抽象为图结构中的局部结构,此时蛋白质复合物预测问题转换为了图结构的局部子图的发现问题。

  然而现有的计算方法通常聚焦于网络中挖掘子图的部分,将复合物预测当成加权或无权网络的密集子图发现问题,而缺少对挖掘出的子图做评价的部分。蛋白质复合物的预测到对预测结果的评价之间存在改进的空间。不同复合物预测方法可以得到大量的复合物预测结果,由于预测方法本身存在的缺陷,预测结果中存在部分不符合复合物形成规律的样本,而预测方法无法精确的识别并剔除这部分复合物。
  
  针对该问题,本文基于预测后复合物样本的评价与筛选模型研究(简称后筛选)设计可行且有效的模型框架,基于多种图卷积神经网络及其变种实现了复合物特征子图的评分模型。

  本文具体进行了如下研究:

  1)构建融合结点特征和邻边特征的蛋白质相互作用网络。通过生物学上的多种特征提取方法,以及GAE、Deepwalk等网络嵌入方法,本文得到了兼具结点特征和邻边特征的$PIN$网络结构。

  2)复合物子图数据集构建。基于邻居相似性融合多个标准集构建了正样本数据集;基于COACH算法的结果构建中间样本数据集;提出了改进的随机算法构建负样本数据集。在特征$PIN$中抽取特征子图作为训练样本。

  3)复合物分类模型研究。在具有特征网络的前提下,本文研究了多种融合特征的复合物分类模型,提出了针对结点特征的图卷积模型、针对邻居特征的EdgeConv模型以及实现特征融合的消息传递网络模型。

  4)相关对比实验及验证。对于每一个模型本文进行了多项实验验证模型的有效性,$PPI$网络包括DIP网络和Biogrid网络,并在Dpclus、ipca等方法上验证筛选结果,验证的指标包括f1值和复合物预测邻域的融合指标。

\end{cabstract}

\ckeywords{蛋白质复合物,监督学习,图神经网络,生物信息学,图分类模型,后处理}

\begin{eabstract}
  TODO
\end{eabstract}

\ekeywords{TODO}
