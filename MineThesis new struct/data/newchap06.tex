\chapter{结论与展望}
\label{chapter:SummaryAndForward}

\section{总结}
\label{section:allsummary}

本文总结了现有蛋白质复合物预测方法的局限性,包括无法充分利用蛋白质复合物的先验知识,不能充分融合蛋白质特征和蛋白质相互作用特征,预测结果准确度较低等等。针对现有方法的不足,本文提出了蛋白质复合物筛选模型,利用已知的蛋白质复合物和蛋白质复合物样本生成算法构成特征子图数据集,训练蛋白质复合物评价模型,对待筛选复合物预测结果进行评分与过滤,从而提高预测结果评价指标。

本文具体完成了如下的研究。

1)本文构建了具有结点特征和邻边特征的蛋白质相互作用网络,即特征互作网络。在蛋白质相互作用网络中,基于图自编码器和深度随机游走嵌入结点特征,表达了蛋白质的全局拓扑信息;基于蛋白质生物数据的相似性挖掘算法嵌入多维度邻边特征,表达了蛋白质功能特性。

基于特征互作网络,本文提出了特征子图的概念并构造了蛋白质复合物特征子图数据集。为了有效的学习蛋白质复合物的生成规律,本文结合特征互作网络将蛋白质复合物原始样本映射为对应的特征子图,从而将不可学习的蛋白质集合数据转换为可学习的结构化数据。本文构建了特征子图训练数据集和待筛选数据集,其中训练数据集包括基于已知但蛋白质融合的正样本数据集、基于COACH算法生成的中间样本数据集、基于受限随机游走的负样本数据集,待筛选数据集为基于核心附属蛋白质复合物预测算法的待筛选样本数据集。

2)本文提出了基于图卷积神经网络的复合物筛选模型。模型从拓扑数据出发,基于图卷积神经网络的方法提取了深层次的蛋白质结点特征。基于图自编码器和深度随机游走算法,该模型在蛋白质互作网络中嵌入具有全局特性的结点特征。在蛋白质特征子图中,基于图神经网络的方法将结点特征进行动态融合与嵌入转换。实验阶段,该模型与无全局特征的模型、基于子图拓扑统计特征模型进行了对比与分析,表明融合全局特征的图卷积神经网络方法的有效性。
  
3)本文提出了基于邻边卷积的复合物筛选模型。该模型充分的嵌入了生物数据,基于邻边卷积算法融合生物数据,解决了已有算法中蛋白质复合物生物数据抽象程度较低的问题。本章在蛋白质互作网络的邻边中嵌入了三种蛋白质关联相似性特征,包括GO注释特征、拓扑域特征和亚细胞定位特征。在实验中该模型对比了无邻边特征情况下的邻边卷积复合物筛选模型,实验结果表明生物数据邻边嵌入以及邻边卷积融合方法的有效性。

4)本文提出了基于点边消息传递网络的复合物筛选模型。针对蛋白质全局拓扑特征和蛋白质生物相似性特征的融合,本文提出了基于点边消息传递网络的复合物筛选模型。该模型采用消息传递的更新方法在复合物特征子图中交替更新结点特征和邻边特征,实现了全局拓扑特征与生物功能特征的动态更新和深度融合。实验中该模型对比了基于图卷积神经网络筛选模型和基于邻边卷积筛选模型,在多个对比实验中点边消息传递模型对评价指标的提升达到了最优,表明特征融合模型能更好的挖掘蛋白质复合物的形成规律。


\section{展望}
\label{section:forward}

本文提出了特征蛋白质互作网络构建,复合物特征样本集集以蛋白质复合物的评分模型。在此基础上,还有一部分研究可以进一步展开。

1)动态蛋白质复合物子图样本生成。细胞是构成生物体的基本结构,是代谢和功能的基本单元,具有动态性。蛋白质相互作用网络会随着生物进程动态变化,因此构建动态蛋白质相互作用网络更符合研究需求。同时在网络中动态的生成蛋白质复合物样本可以更深层次的研究蛋白质复合物的生成规律。然而动态网络的构建以及动态子图的生成均具有一定的挑战性,也是该领域的研究热点和难点。

2)融合多源的蛋白质特征。本文融合了基础的蛋白质特征和GO注释、拓扑域等相似性特征。蛋白质序列数据和蛋白质的三维结构数据是蛋白质特征的直观的体现,序列数据和三维结构的研究也是蛋白质研究领域的重点。如何将蛋白质序列数据、三维结构数据和蛋白质复合物预测问题结合起来仍然需要进一步研究。

