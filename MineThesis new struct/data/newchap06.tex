\chapter{结论与展望}
\label{chapter:SummaryAndForward}

\section{总结}
\label{section:allsummary}

生物学上蛋白质复合物的发现与研究对细胞组成、药物发现等研究至关重要。蛋白质复合物可以通过生物实验的方法发现,但是生物实验成本较高、周期较长,无法满足后续的研究需求。虽然计算方法可以提供大量的候选复合物,但是这些候选样本通常准确度较低。
本文使用了蛋白质复合物筛选模型提升了蛋白质复合物的预测效果,在多个数据集,多种方法和多个评价指标下得到了验证,为蛋白质复合物预测的提升提供了另一种研究的可能性。

本文具体完成了以下的研究。

构建蛋白质特征互作网络,本文构建了特征蛋白质复合物网络来解决解决生物学特征丢失问题,特征网络中直接在网络构件上保留了相应的特征,其中结点特征包括网络编码特征、蛋白质自有特征,邻边特征包含了丰富的蛋白质互作特征,包括GO注释相似性、拓扑域相似性及亚细胞定位。

在特征蛋白质复合物网络的基础上,本文利用多个标准集构建了正样本数据集;基于COACH算法的结果构建中间样本数据集;提出了改进的随机算法构建负样本数据集。在$PIN$中抽取复合物子图作为训练样本,对整体数据集中的样本进行了分类和评分,为后续分类模型训练提供数据基础。同时,该数据集生成方法也可以支持后续其他可能的复合物分类或回归研究。

本文提出了融合特征的复合物筛选模型,包括融合结点,融合邻边以及交叉融合的模型。模型可以完成已挖掘出的子图的辨别工作,通过预测样本筛选可以提高预测样本的精确度以及多项其他评价指标。同时模型采用了图卷积神经网络和消息传递网络的方法,动态的融合了蛋白质复合物中的蛋白质特征以及蛋白质相互作用特征,极大的提高了筛选模型的筛选能力。本文进行了多个网络数据集和多种挖掘算法的验证实验,证明了筛选模型的有效性。在DIP网络和Biogrid网络上进行分别训练了各个预测模型,并在Dpclus、ipca等方法上验证筛选结果,验证的指标包括f1值和复合物预测邻域的融合指标。结果表明,本文提出的特征融合模型在DIP和Biogrid网络中均取得了高于筛选前样本的评价指标,证明了方法的有效性。

\section{展望}
\label{section:forward}

