\tongjisetup{
  %******************************
  % 注意:
  %   1. 配置里面不要出现空行
  %   2. 不需要的配置信息可以删除
  %******************************
  %
  %=====
  % 秘级
  %=====
  secretlevel={保密},
  secretyear={2},
  % doctor={0}
  %
  %=========
  % 中文信息
  %=========
  % 题目过长可以换行(推荐手动加入换行符,这样就可以控制换行的地方啦)。
  ctitle={蛋白质复合物筛选模型研究},
  cheadingtitle={蛋白质复合物筛选模型研究},    %用于页眉的标题,不要换行
  cauthor={高盼},
  studentnumber={1830801},
  cmajorfirst={工学},
  cmajorsecond={计算机科学与技术},
  cdepartment={电子与信息工程学院},
  csupervisor={关佶红 教授},
  % 如果没有副指导老师或者校外指导老师,把{}中内容留空即可,或者直接注释掉。
  % cassosupervisor={裴刚 教授~(校外)}, % 副指导老师
  % 日期自动使用当前时间,若需手动指定,按如下方式修改:
  % cdate={\zhdigits{2018}年\zhnumber{11}月},
  % 没有基金的话就注释掉吧。
  cfunds={(国家自然基金项目(No.61772367,U1936205)资助},
  %
  %=========
  % 英文信息
  %=========
  etitle={Study on protein complex screening model},
  eauthor={Pan Gao},
  emajorfirst={Engineering},
  emajorsecond={Computer Science and Technology},
  edepartment={School of Electronic and Information Engineering},
  % 日期自动使用当前时间,若需手动指定,按如下方式修改:
  % edate={November,\ 2018},
  efunds={(Supported by the Natural Science Foundation of China(No.61772367,U1936205))},    
  esupervisor={Prof. JiHong Guan}
  % eassosupervisor={Prof. Gang Pei (XiaoWai)}
  % }
}
% 定义中英文摘要和关键字
\begin{cabstract}
  蛋白质复合物是蛋白质相互结合完成某一项生物功能的集合。生物学上蛋白质复合物的识别与研究对细胞组成分析、药物预测等至关重要。生物实验的方法可以识别蛋白质复合物,但其成本较高、周期较长,无法满足大规模数据时代的研究需求。
  现有的蛋白质复合物预测算法主要是基于计算的算法,将蛋白质之间广泛的相互作用抽象成图,蛋白质复合物抽象为图中的局部结构,此时蛋白质复合物预测问题转换为局部子图发现问题。复合物预测算法可以从互作网络中挖掘出大量的局部子图样本,但是由于预测算法本身的局限性,预测结果中会存在部分不符合复合物形成规律的样本。基于计算的复合物预测算法不具有对其预测结果的评价能力,无法识别并剔除这部分样本。

  为了识别和剔除这部分对预测结果具有干扰性的样本,本文提出了蛋白质复合物筛选模型,构建了融合结点特征和邻边特征的蛋白质相互作用网络,并进一步构造了复合物特征子图样本集。本文提出了多种复合物筛选模型并针对每个模型进行了模型训练以及复合物筛选验证,复合物筛选模型包括基于图卷积的模型、基于邻边卷积的模型和基于点边消息传递网络的模型。

  本文的研究工作及贡献:

  1)融合结点特征和邻边特征的蛋白质相互作用网络(简称特征互作网络)。基于蛋白质互作数据构建了蛋白质相互作用网络,基于图自编码器、深度随机游走等网络嵌入方法为网络添加了结点特征,基于生物学上的多种相似性特征提取方法为网络添加了邻边特征,最终构建出特征互作网络。

  构造了多种类别的蛋白质复合物特征子图样本集。原始蛋白质复合物数据集以蛋白质集合的形式存在,无法满足样本后筛选工作的要求。本文基于特征子图提取方法,在特征互作网络中构建对应蛋白质复合物数据集的特征子图样本集,包括基于邻居相似性融合多个标准集构建的正样本数据集;基于COACH算法构建的中间样本数据集;基于改进的随机算法构建的负样本数据集;基于核心附属结构的复合物预测算法构建的待筛选数据集。

  2)基于图卷积神经网络的复合物筛选模型。从拓扑数据出发,本文提出了基于图卷积神经网络的复合物筛选模型。该模型基于图自编码器和深度随机游走嵌入得到蛋白质全局特征,并依据蛋白质复合物子图的拓扑结构,基于图神经网络对蛋白质全局特征进行深度融合。本文将该模型与无全局特征的模型、基于拓扑统计特征的模型进行了对比与分析,结果表明了融合全局特征的图卷积神经网络方法的有效性。
  
  3)基于邻边卷积的复合物筛选模型。从生物数据出发,针对蛋白质复合物生物数据抽象程度较低的问题,本文提出了基于邻边卷积的复合物筛选模型。该模型在蛋白质互作网络中嵌入了表达蛋白质功能关联的多种相似性数据,作为互作网络邻边特征,并基于邻边卷积的方法将子图拓扑和邻边特征进行融合。实验对比了有无邻边特征情况下的邻边卷积模型,结果显示了带邻边特征的邻边卷积模型的有效性。

  4)基于点边消息传递网络的复合物筛选模型。从特征融合角度出发,本文提出了基于点边消息传递网络的复合物筛选模型。该模型保留了表示全局拓扑特性的结点特征和表示生物功能特性的邻边特征,采用消息传递的更新方法在蛋白质复合物局部拓扑结构中同时更新结点特征和邻边特征,实现了全局拓扑特征与生物功能特征的动态更新和深度融合。结果表明基于点边消息传递网络地复合物筛选模型能更好地挖掘蛋白质复合物的形成规律,在多个实验中其评价指标的提升达到了最优。

\end{cabstract}

\ckeywords{蛋白质复合物预测;图神经网络;蛋白质相互作用网络;子图分类;图嵌入}

\begin{eabstract}
  A protein complex is a collection of proteins that bind to each other to accomplish a biological function. The identification and research of protein complexes in biology are very important for cell composition analysis, drug prediction and so on. The method of biological experiment can identify protein complex, but its cost is high and the cycle is long, which can not meet the research demand of large-scale data age.
  The existing protein complex prediction algorithm is mainly based on the calculation algorithm, abstracting the wide interaction between proteins into a diagram, protein complex abstraction into the local structure of the diagram, at which time the protein complex prediction problem is transformed into a local sub-graph to find the problem. The composite prediction algorithm can extract a large number of local subgraph samples from the mutual network, but due to the limitations of the prediction algorithm itself, there will be some samples in the prediction results that do not conform to the formation law of the compound. The calculation-based compound prediction algorithm does not have the ability to evaluate its prediction results, and cannot identify and reject this part of the sample.

  In order to identify and reject the samples which are disturbing to the prediction results, this paper presents a protein complex screening model, constructs a protein interaction network that combines node features and adjacent features, and further constructs a sample set of composite feature sub-maps. In this paper, a variety of composite filtering models are proposed and model training and composite filtering validation are carried out for each model, including models based on tortogram remnants, models based on adjacent curly volume, and models based on point edge messaging networks.

  The research work and contribution of this paper:

  1) A network of protein interactions (referred to as a network of features) that combines node features with adjacent features. Based on the protein interoperability data, the protein interaction network is constructed, the node characteristics are added to the network based on the network embedding methods such as graph self-encoder and deep random walk, and the adjacent features are added to the network based on various similarity characteristic extraction methods in biology, and the characteristic interoperability network is finally constructed.

  A sample set of sub-maps of protein complex features in various categories is constructed. The original protein complex dataset exists in the form of a protein collection and cannot meet the requirements of post-sample screening. Based on the feature sub-graph extraction method, this paper constructs the characteristic sub-map sample set of the corresponding protein complex data set in the feature interoperability network, including the positive sample data set built on the basis of multiple standard sets of neighbor similarity fusion, the intermediate sample data set based on THECH algorithm, the negative sample data set based on the improved random algorithm, and the pending filtering data set based on the composite prediction algorithm of the core satellite structure.


  2) A composite screening model based on a refride neural network. Based on the topological data, this paper presents a composite screening model based on the refride neural network. The model obtains the global characteristics of proteins based on graph self-encoder and deep random walk embedding, and based on the topology of protein complex subpic, the global characteristics of proteins are deeply fused based on the graph neural network. In this paper, the model is compared and analyzed with the model without global characteristics and based on topological statistical features, and the results show the effectiveness of the topographical neural network method of fusing global features.

  3) A composite filtering model based on neighboring coils. Based on the biological data, in view of the low abstraction of the biological data of protein complexes, this paper puts forward a composite screening model based on the neighboring reftric. The model embeds a variety of similarity data expressing protein function association in the protein interoperability network, as a mutual network neighbor feature, and fuses the subgraph topology and neighbor feature based on the method of neighbor converse. The experiment compares the neighborhood remnoms model with or without neighbor features, and the results show the validity of the neighboring reuter model with neighbor features.

  4)A composite filtering model based on a point-edge messaging network. From the point of view of feature fusion, this paper proposes a composite filtering model based on point-edge messaging network. The model retains the node characteristics representing the global topological characteristics and the neighbor characteristics representing the biological functional characteristics, and uses the method of message delivery to update the node features and adjacent features in the local topology of the protein complex at the same time, and realizes the dynamic update and deep fusion of the global topological features and biological functional features. The results show that the formation law of protein complex can be better excavated based on the point-edge message transmission network composite screening model, and the improvement of its evaluation index has reached the optimal level in many experiments.

\end{eabstract}

\ekeywords{Protein complex prediction; graph neural network; protein interaction network; sub-graph classification; graph embedding}
