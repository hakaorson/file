\chapter{结论与展望}
\label{chapter:SummaryAndForward}

\section{总结}
\label{section:allsummary}

现有蛋白质复合物预测问题中存在一定的局限性,现有方法无法充分利用已验证蛋白质复合物的先验知识,不能充分融合蛋白质特征和蛋白质相互作用特征,预测结果的准确度具有提升的空间。针对这些不足,本文提出了了蛋白质复合物筛选模型框架,该框架利用已知的蛋白质复合物数据以及其他相关数据训练蛋白质复合物评价模型,基于评分模型对已有蛋白质复合物预测算法的预测结果进行评分与筛选,从而提高预测结果的准确性。

本文具体完成了如下的研究。

1)蛋白质特征与蛋白质相互作用特征的嵌入。本文构建具有特结点特征和邻边特征征的蛋白质相互作用网络。基于图自编码器和深度随机游走嵌入方法嵌入结点特征向量,使得结点特征包含了一定的但蛋白质相互作用网络全局拓扑信息,同时基于部分蛋白质自有特征进行了结点嵌入。基于生物数据基础上的多种相似性计算方法嵌入邻边特征向量,使得邻边特征包含了丰富的蛋白质互作信息。

2)构造充分利用先验知识的蛋白质复合物数据集。为了学习蛋白质复合物的生成规律,本文结合特征蛋白质相互作用网络将蛋白质复合物映射为对应的特征子图。构建了基于已知但蛋白质融合的正样本特征子图、基于COACH算法生成的中间样本特征子图、基于受限随机游走的负样本特征子图以及基于核心附属蛋白质复合物预测算法的待筛选特征子图。

3)基于多种特征融合方法的蛋白质复合物评价模型以及相关实验。本文针对结点特征、邻边特征和融合特征分别提出了对应的模型。基于结点特征模型提出了基于结点的GCN方法,融合结点的全局拓扑特征进行复合物预测,该模型将蛋白质相互作用网络特征进行了嵌入;基于邻边特征模型提出了基于邻边的GCN方法,将初始的邻边特征转换为结点特征,再基于edge-base的GCN方法对特征进行融合,该模型充分融合了生物特征和拓扑特征,相较于邻边映射方法挖掘了深层次高抽象的邻边特征;基于特征融合模型提出了融合结点特征和邻边特征的MPNN方法,提出了结点特征和邻边特征交替更新的方法,在特征融合同时保留自身关键特征。


\section{展望}
\label{section:forward}

本文提出了特征蛋白质互作网络构建,复合物特征样本集集以蛋白质复合物的评分模型。在此基础上,还有一部分研究可以进一步展开。

1)将蛋白质复合物评分模型作为复合物评价函数,利用强化学习的方法在蛋白质相互作用网络中挖掘蛋白质复合物。

2)本文使用的池化方法是基于平均池化和最大值池化,然而这种全局性质的池化方法可能丢失一部分蛋白质复合物子图的结构信息,后续可以考虑基于Graph Pool为基础的分层池化模型,以进一步在复合物特征表示上提升其拓扑表达能力。

