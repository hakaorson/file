\chapter{结论与展望}
\label{chapter:SummaryAndForward}

\section{总结}
\label{section:allsummary}

生物学上蛋白质复合物的发现与研究对细胞组成、药物发现等研究至关重要。已有的蛋白质复合物预测方法主要思路是将蛋白质之间广泛的相互作用抽象成图结构,蛋白质复合物抽象为图结构中的局部结构,此时蛋白质复合物预测问题转换为了图结构的局部子图的发现问题。
然而这些方法通常聚焦于网络中挖掘子图的部分,将复合物预测当成加权或无权网络的密集子图发现问题,缺少对挖掘出的子图做评价的部分。

针对该问题,本文填补了蛋白质复合物的预测到对预测结果的评价之间存在的改进空间。
本文基于预测后复合物样本的评价与筛选模型研究(简称后筛选)设计可行且有效的模型框架,基于多种图卷积神经网络及其变种实现了复合物特征子图的评分模型。

本文第一章介绍了主流的蛋白质复合物预测算法,并根据存在的问题提出了可改进的空间。对研究背景、研究意义以及本文的主体研究流程做了相关介绍。
介绍了筛选模型的总体框架结构。
第二章阐述了具有结点特征和邻边特征的蛋白质相互作用网络的构建方法,包括GAE和Deepwalk的结点特征嵌入、以及多种基于蛋白质相似性计算的邻边嵌入方法;后续阐述了如何从特征相互作用网络中提取蛋白质复合物子图,以及训练子图样本、待筛选子图样本的生成方法。
第三章从拓扑特征入手,提出了利用结点的拓扑特征并基于结点的图卷积神经网络的复合物筛选模型,挖掘$PIN$网络中结点嵌入对复合物预测的作用。同时提出了本文算法框架中的参数细节,评价指标等等。后续进行了实验并验证模型的有效性。
第四章从特征子图中的邻边相似性出发,探讨了使用生物特征转换为的邻边数据的基础上,复合物筛选模型可达到的效果提升。提出了使用基于邻边卷积的图神经网络方法以及相应的模型,阐述了邻边特征如何初始化结点特征,以及EdgeConv的信息流更新过程。后续进行了相关对比实验并验证模型的有效性。
第五章在融合$PIN$全局特征以及生物相似性特征的情况下,提出了基于消息传递网络的复合物筛选模型。
特征子图具有GAE和Deepwalk特征等结点特征、多种蛋白质相似性嵌入邻居特征,提出了改进的MPNN更新方法,实现了结点与邻边交替融合更新的模型。最后对比了利用结点特征的基于图卷积的模型,利用相似性特征的基于邻边卷积的模型,并进行实验验证。

本文的研究成果如下:

1)构建融合结点特征和邻边特征的蛋白质相互作用网络。通过生物学上的多种特征提取方法,以及GAE、Deepwalk等网络嵌入方法,本文得到了兼具结点特征和邻边特征的$PIN$网络结构。

2)复合物子图数据集构建。基于邻居相似性融合多个标准集构建了正样本数据集;基于COACH算法的结果构建中间样本数据集;提出了改进的随机算法构建负样本数据集。在特征$PIN$中抽取特征子图作为训练样本。该样本集可供后续的复合物分类以及预测算法使用。

3)复合物分类模型研究。提出了蛋白质复合物评分模型并提出了基于提升已有方法预测质量的研究方法。在具有特征网络的前提下,本文研究了多种融合特征的复合物分类模型,提出了针对结点特征的图卷积模型、针对邻居特征的EdgeConv模型以及实现特征融合的消息传递网络模型。
本文首次将图分类模型引入了蛋白质复合物评价中,通过大量的实验验证了$PIN$网络中结点嵌入的有效性,生物数据作为邻边特征的有效性,以及融合方法的有效性。



\section{展望}
\label{section:forward}

本文提出了特征蛋白质互作网络构建,复合物特征样本集集以蛋白质复合物的评分模型。在此基础上,还有一部分研究可以进一步展开。

1)将蛋白质复合物评分模型作为复合物评价函数,利用强化学习的方法在蛋白质相互作用网络中挖掘蛋白质复合物。

2)本文使用的池化方法是基于平均池化和最大值池化,然而这种全局性质的池化方法可能丢失一部分蛋白质复合物子图的结构信息,后续可以考虑基于Graph Pool为基础的分层池化模型,以进一步在复合物特征表示上提升其拓扑表达能力。

