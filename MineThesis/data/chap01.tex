\chapter{绪论}
\label{chapter:intro}

\section{研究背景}
\label{section:background}

生物信息学是研究生物信息的采集、处理、存储、传播和解释等各方面的学科。
生物信息学可以帮助人类从海量的生物数据中挖掘内部的生理过程规律,从而指导进一步的生物学研究。在规模化实验大力发展和海量数据爆发的如今,如何有效的利用生物数据变得越来越中了。
生物信息学分为三个主要的发展阶段,前基因组时代主要建立了各种生物数据库和序列比较算法、基因组时代进行了大规模的基因测序以及目前所处的后基因组时代。
后基因组时代研究重心以生物数据分析为主,并且挖掘的层次逐渐深入。已经从对基因组直接的结构的研究逐渐转向对基因功能的研究,其中的主要侧重点包括基因组学、转录组学以及蛋白质组学等\cite{Helms2008Principles}。

蛋白质是生物细胞和组织的重要组成部分,是生命体的物质基础,也是遗传信息的直接表达手段,涉及生物体载体、免疫、激素等方方面面。蛋白质分子深度参与了组织的构成与修复、生理功能的调节和能量的供给。
蛋白质组学\cite{Schubert2017Quantitative}是在蛋白质表达层面研究生理生化功能为主的一门学科,目的是揭示蛋白质的基本生命活动规律,其中研究主要关注蛋白质结构、蛋白质丰度、蛋白质修饰以及蛋白质相互作用。

蛋自质在细胞活动中发挥着巨大的作用。但是在多数情况下单个蛋自质无法独立的执行生物功能,只有构成蛋白质复合物,才能有效的参与到细胞活动中\cite{Gavin2018Functional}。因此蛋白质复合物的结构、功能及形成方式的研究就显得尤为重要。图\ref{fig:swr1_complex}是具有染色质重塑功能的酵母SWR1复合物。
\begin{figure}[htbp]
  \centering
  \includegraphics{SWR1_complex}
  \caption{酵母菌SWR1复合物}
  \label{fig:swr1_complex}
\end{figure}
生物实验中检测蛋白质复合物主要通过串联亲和纯化与质谱分析\cite{Rigaut1999A}、酵母双杂交\cite{Li1993Identification}两种技术对蛋白质复合物进行分离和鉴定。串联亲和纯化与质谱分析通过靶蛋白标定以及自然条件下亲和纯化获取可能的蛋白质复合物,再使用质谱分析进行鉴定。酵母双杂交技术利用转录调控因子中的组件特征研究蛋白质之间的相互作用关系。虽然基于实验测定的方法具有生物学上的可解释性,但是生物实验往往条件困难、实验步骤多且成本昂贵,无法满足快速增长的研究需求。

蛋白质与生理环境存在广泛的相互作用,蛋白质复杂功能的实现同蛋白质之间、DNA与蛋白质、RNA与蛋白质的相互作用密切相关,蛋白质复合物正是一组强相关的蛋白质组合共同作用的结果。随着生物信息学的发展以及高通量技术的发展,蛋白质相互作用关系(Protein-Protein Interaction,$PPI$,后简称为互作关系)得到了大量的补充,促成了大规模互作网络的构建\cite{Butland2005Interaction},即蛋白质相互作用网络(Protein-Protein Interaction Network,$PIN$)。图\ref{fig:ppi}为酵母菌蛋白质相互作用网络。
\begin{figure}[htbp]
  \centering
  \includegraphics{ppi}
  \caption{酵母菌蛋白质相互作用网络}
  \label{fig:ppi}
\end{figure}
已知的蛋白质复合物可以视作互作网络里面的一系列子网络,因此利用图论和计算模型学习子网络的分布模式,即可在互作网络中发掘潜在的蛋白质复合物\cite{2001Protein}。利用图论的方法,蛋白质互作网络可以转换为无向图~$G=(V,E)$。其中$V$为图中结点的集合,表示所有的蛋白质,$E$为图中邻边的集合,表示所有的蛋白质互作关系。$PIN$转换为图结构之后,图论上的计算模型和深度学习方法就可以迁移到蛋白质复合物的研究中。2002年,Tong\cite{2001A}等人提出了密集子图的假说,蛋白质复合物在$PIN$是密集链接的子图,而与其他的蛋白质连接相对稀疏。在$PIN$中预测蛋白质复合物的问题就转换为了密集子图的挖掘问题,逐渐发展出了利用$PIN$和计算方法识别蛋白质复合物的理论。

\section{国内外研究现状}
\label{section:research}

现有的基于计算方法的蛋白质复合物预测方法主要分为五类:基于网络结构的图聚类方法、融合生物信息的图聚类方法、核心附属扩展方法、动态网络方法以及监督学习方法。以下分别对这五类方法的研究现状做简单的介绍。

\subsection{基于网络结构的图聚类方法}
\label{section:Topology}
\subsection{融合生物信息的图聚类方法}
\label{section:appendBiology}
\subsection{核心附属扩展方法}
\label{section:CoreAppend}
\subsection{动态网络方法}
\label{section:Dynamic}
\subsection{监督学习方法}
\label{section:Supervision}
\subsection{方法总结}
\label{section:researchSummary}

\section{研究动机和贡献}
\label{section:motivation}

\section{论文组织}
\label{section:organization}