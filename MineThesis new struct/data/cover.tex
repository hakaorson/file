\tongjisetup{
  %******************************
  % 注意:
  %   1. 配置里面不要出现空行
  %   2. 不需要的配置信息可以删除
  %******************************
  %
  %=====
  % 秘级
  %=====
  secretlevel={保密},
  secretyear={2},
  % doctor={0}
  %
  %=========
  % 中文信息
  %=========
  % 题目过长可以换行(推荐手动加入换行符,这样就可以控制换行的地方啦)。
  ctitle={蛋白质复合物筛选模型研究},
  cheadingtitle={蛋白质复合物筛选模型研究},    %用于页眉的标题,不要换行
  cauthor={高盼},
  studentnumber={1830801},
  cmajorfirst={工学},
  cmajorsecond={计算机科学与技术},
  cdepartment={电子与信息工程学院},
  csupervisor={关佶红 教授},
  % 如果没有副指导老师或者校外指导老师,把{}中内容留空即可,或者直接注释掉。
  % cassosupervisor={裴刚 教授~(校外)}, % 副指导老师
  % 日期自动使用当前时间,若需手动指定,按如下方式修改:
  % cdate={\zhdigits{2018}年\zhnumber{11}月},
  % 没有基金的话就注释掉吧。
  % cfunds={(本论文由我要努力想办法撑到两行的著名国家杰出青年基金 (No.123456789) 支持)},
  %
  %=========
  % 英文信息
  %=========
  etitle={English Title},
  eauthor={GAO PAN},
  emajorfirst={TODO},
  emajorsecond={TODO},
  edepartment={TODO},
  % 日期自动使用当前时间,若需手动指定,按如下方式修改:
  % edate={November,\ 2018},
  % efunds={(Supported by the Natural Science Foundation of China for\\ Distinguished Young Scholars, Grant No.123456789)},    
  esupervisor={TODO}
  % eassosupervisor={Prof. Gang Pei (XiaoWai)}
  % }
}
% 定义中英文摘要和关键字
\begin{cabstract}
  蛋白质复合物是蛋白质相互结合,完成某一项生物功能的集合。生物学上蛋白质复合物的发现与研究对细胞组成、药物发现等研究至关重要。蛋白质复合物可以通过生物实验的方法发现,但是生物实验成本较高、周期较长,无法满足后续的研究需求。
  
  随着计算理论的发展,蛋白质复合物发现邻域出现了基于计算的方法,将蛋白质之间广泛的相互作用抽象成图结构,蛋白质复合物抽象为图结构中的局部结构,此时蛋白质复合物预测问题转换为了图结构的局部子图的发现问题。

  然而现有的计算方法通常聚焦于网络中挖掘子图的部分,而缺少对挖掘出的子图做评价的部分。已有的算法通常将复合物预测当成加权或无权网络的密集子图发现问题,蛋白质丰富的生物学特征未考虑或仅仅硬编码为连边权重。

  针对已有的问题,本文构建了特征蛋白质复合物网络来解决解决生物学特征丢失问题,特征网络中直接在网络构件上保留了相应的特征,其中结点特征包括网络编码特征、蛋白质自有特征,邻边特征包含了丰富的蛋白质互作特征,包括GO注释相似性、拓扑域相似性及亚细胞定位。
  
  在特征蛋白质复合物网络的基础上,本文提出了融合特征的复合物筛选模型,包括融合结点,融合邻边以及交叉融合的模型。模型可以完成已挖掘出的子图的辨别工作,通过预测样本筛选可以提高预测样本的精确度以及多项其他评价指标。同时模型采用了图卷积神经网络和消息传递网络的方法,动态的融合了蛋白质复合物中的蛋白质特征以及蛋白质相互作用特征,极大的提高了筛选模型的筛选能力。
  
  最后本文进行了多个网络数据集和多种挖掘算法的验证实验,证明了筛选模型的有效性。
\end{cabstract}

\ckeywords{蛋白质复合物,监督学习,图神经网络,生物信息学,图分类模型,后处理}

\begin{eabstract}
  TODO
\end{eabstract}

\ekeywords{TODO}
