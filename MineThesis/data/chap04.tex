\chapter{实验结果与分析}
\label{chapter:resultAndOther}
两个网络上计算,因为比较经典
额外的三个方法测试,
只有最后最优模型才在四个网络上全跑,但是只是写一个同时,然后放一个实验结果说明有效就可以了
\section{实验设计}
\label{section:workdesign}
\subsection{模型细节}
\label{subsection:trainModel}

由于训练数据的不平衡性,训练模型之前,数据进行了重采样处理,保持各分类训练样本相等。按照$8:2$的比率分割各分类样本,形成训练集和测试集。
本文采用0.001的学习率,batch大小设置为32,隐层维度为128,模型使用了两层的GCN网络以及两层的分类器。

激活函数和优化器TODO

每一个训练复合物样本都具有复合物类别以及相似性评分两项数据,针对这两项属性,本文实验中分别设置了两项损失函数,使用交叉熵损失(Cross Entropy Loss)计算复合物类别分类损失,使用BCE损失(Binary Cross Entropy Loss)计算复合物得分回归损失。由于两项损失在数值上的差异,设置了相应的损失平衡系数。具体损失计算如下所示。
\begin{equation}
    \label{equ:loss}
    loss=CELoss(PL,TL)+\alpha \cdot BCELoss(PS,TS)
\end{equation}
其中,$PL$为预测类别,$TL$为真实类别,$PS$为预测评分,$TS$为真实评分,$\alpha$为损失平衡系数。

模型测试阶段,本文将复合物生成算法A在$PIN$网络中运行得到算法A预测的复合物$Complexes_A$,用腹黑谁分类模型筛选所有预测结果,其中如果预测样本在筛选模型中被分类为真样本或者评分高于0.25时,预测样本即可通过筛选。最后所有通过筛选的复合物形成复合物集合$Complexes_A'$。最后使用F1值和PPV指标评估$Complexes_A$和$Complexes_A'$的复合物质量。
\subsection{评估指标}
\label{subsection:netrix}
评价复合物预测结果的常用指标为精准率(precision)、召回率(recall)和F1值(F1-score)。预测复合物和标准复合物计算邻居相似性(NA-similarity)\ref{equ:compComplexSim:NA},计算方式如下所示。
\begin{equation}
    \label{equ:compComplexSim:NA}
    NA_{(\mathcal{P} ,\mathcal{Q} )} = \frac{{\left\lvert V_{\mathcal{P}} \cap V_{\mathcal{Q}}\right\rvert}^2 }{{\left\lvert V_{\mathcal{P}} \right\rvert}\cdot  {\left\lvert V_{\mathcal{Q}} \right\rvert}}
\end{equation}
其中$\mathcal{P}=(V_{\mathcal{P}} ,E_{\mathcal{P}})$是预测复合物,$V_{\mathcal{P}}$为复合物中的蛋白质,$E_{\mathcal{P}}$为复合物中的相互作用,$\mathcal{Q}=(V_{\mathcal{Q}} ,E_{\mathcal{Q}})$是标准复合物。

按照一般性标准,邻居相似性大于0.25时,预测复合物和标准复合物具有匹配性,表示复合物预测成功。
精准率计算如公式\ref{equ:precision}所示。
其中$PC$为预测复合物集合,$M_{PC}$为预测复合物中具有匹配性的所有复合物集合。精准率表示预测复合物中,预测成功的样本所占比重,衡量预测结果的“查准率”。
召回率段计算如公式\ref{equ:recall}所示。
其中$BC$为标准复合物集合,$M_{BC}$为标准复合物中被预测复合物匹配的集合。召回率表示标准复合物中,能被预测复合物匹配的样本所占的比重,衡量预测结果的“查全率”。
F1值综合考虑查准率和查全率,计算如公式\ref{equ:f1}所示。
从定义看出,当查准率和查全率中有一项指标偏小时,F1值就会偏小。
\begin{equation}
    \label{equ:precision}
    precision=\frac{\left\lvert M_{PC}\right\rvert }{\left\lvert PC\right\rvert }
\end{equation}
\begin{equation}
    \label{equ:recall}
    recall=\frac{\left\lvert M_{BC}\right\rvert }{\left\lvert BC\right\rvert }
\end{equation}
\begin{equation}
    \label{equ:f1}
    f-score=\frac{2\times precision\times recall}{precision + recall }
\end{equation}

除此之外,评价复合物预测质量还可采用PPV指标(Cluster-wise Positive Predictive Value)\cite{shi_protein_2011},其具体计算方式如式\ref{equ:PPV}所示,其中${j| 1,2,\dots,m }$表示所有的预测复合物,${i| 1,2,\dots,n }$表示所有的标准复合物,$t_{ij}$表示两个复合物$i,j$之间共同蛋白质的数量。PPV值越高,表示预测复合物和标准复合物的平均重合度越高。

\begin{equation}
    \label{equ:PPV}
    PPV=\frac{\sum_{j = 1}^{m} \max_{i=1}^{n} t_{ij}}{\sum_{i = 1}^{n} \sum_{j = 1}^{m}   t_{ij}}
\end{equation}

\section{生物特征的影响}E
\label{section:biofeatAnasys}
先比较边卷积模型和图特征的模型,在分类问题上的表现
然后是边卷积模型,融合特征边卷积模型总体的表现


两个网络上计算
额外的三个
\section{全局特征的影响}
\label{section:globalStructfeatAnasys}
两个网络上计算
\section{特征融合模型的实验结果分析}
\label{section:fusionfeatAnasys}
同时加上部分蛋白质的生物特征

\section{图读出方法的影响}
\label{section:gpoolAnasys}
得出最佳模型,并和刘晓霞做比较,由于什么的限制,只能在我的数据上跑刘晓霞的方法

\section{实验总结与分析}
\label{section:resultSummary}